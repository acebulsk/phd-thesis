% abstract.tex
\thesisabstract{%
Interception and ablation of snow in needleleaf canopies regulates the quantity, timing, and phase of precipitation that reaches the ground in cold region forests. Sparse observations have limited understanding of these processes and hindered their representation in hydrological and land-surface models. Consequently, simulations of subcanopy snow accumulation have shown variable accuracy across different climates and canopy structures, and diagnosing the causative processes that partition snowfall in needleleaf forests remains difficult. This study aims to better understand snow interception and canopy snow ablation processes, which act collectively to govern subcanopy snow accumulation in forested environments. To achieve this goal a comprehensive review of the literature was conducted, along with in-situ observations of forest-snow processes and subcanopy snow accumulation to evaluate the theories underlying existing parameterisations, derive new relationships, and validate a newly proposed model. 

The results from the literature review showed that existing parameterisations for snow interception and canopy snow ablation have incomplete and potentially duplicated process representation. For example, existing studies show that existing snow interception parameterisations may overemphasize the role of increasing canopy snow load increasing throughfall of snow in the initial interception process. This association between snow load and throughfall has been attributed to either reduced canopy coverage due to branch bending [@Schmidt1990; @Schmidt1991] and/or increased rates of unloading. @Staines2023 showed using novel aerial lidar based observations of canopy density and throughfall found limited influence of snow load on canopy coverage due to branch bending. Therefore, if the relationship of initial interception with snow load is not resulting from canopy density changes, perhaps it may be better represented using the unloading parameterisation alone to avoid double counting this process. Moreover, the methodologies used to parameterise non-melt induced canopy snow unloading routines rely on indirect measurements and should be assessed using more direct measurements. While many canopy snow ablation processes occur simutainiously using a hybrid modelled-observed processes to help isolate individual canopy snow processes has not been utilised to help understand canopy snow processes. 

New observations of initial snow interception, collected when ablation processes were minimal, showed that canopy density is the primary factor influencing subcanopy snow throughfall. Contrary to existing theories, no relationships were found with canopy snow load or air temperature. Canopy density was best represented by a snow-leaf contact area metric, which accounts for increasing contact area with more horizontal snowfall trajectories. This metric was highly sensitivity to wind speed in sparse canopies, increasing by up to an order of magnitude with a 1 m s$^{-1}$ wind speed. A new parameterisation was developed that calculates throughfall as a function of snowfall and snow-leaf contact area based on these observations. This approach is consistent with some rainfall interception studies, which also separate canopy loading and ablation processes, and calculate interception as a function of canopy cover.

Observations showed that canopy snow unloading was strongly associated with snow load, wind shear stress, and canopy snowmelt, but not with air temperature or sublimation. A new canopy snow ablation model was developed based on these associations and their impact on the canopy snow energy and mass balance. The model improved simulation of canopy snow load relative to previous approaches, especially for melt- and wind-dominated ablation events. The improved performance in representing canopy snow load compared to existing models results from including energy balance-based melt and dry snow unloading relationships with snow load and wind shear stress. The inclusion of both melt and dry snow unloading processes in the new model also led to more accurate partitioning of snowfall to the atmosphere versus the ground compared to existing approaches across a wide range of meteorologies.

Validation of the revised canopy snow mass and energy balance was conducted at four needleleaf forests characterised by differing tree species, canopy structure, and meteorological conditions. The new physically based approach reduced error in simulating subcanopy snow water equivalent (SWE) compared to an existing model. Improved process representation also enabled clearer process diagnosis on the influence of vegetation on snow accumulation. At two cold, low-snowfall sites, roughly half of annual snowfall was lost to the atmosphere via sublimation of intercepted snow. A cold wind-exposed site with higher annual snowfall had increased unloading rates and reduced atmospheric losses. In contrast, at a temperate-maritime site, nearly half of annual snowfall melted within the canopy, and combined with melt-induced unloading led to the lowest fraction of snowfall lost through canopy snow sublimation.

This research identified key limitations in existing models using in-situ process-level observations of snow interception and ablation. New parameterisations of the canopy snow mass and energy balance are introduced to more explicitly represent the processes that govern snowfall partitioning in needleleaf canopies. This new approach led to more accurate simulations of canopy snow load and subcanopy SWE across a broad range of meteorologies and forest types and also improved process diagnosis, though further validation is required across a diverse range of sites.
}
