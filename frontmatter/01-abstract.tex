% abstract.tex
\thesisabstract{%
 Partitioning of snowfall by forest canopies, which occurs over 23\% of the global land mass, represents an important process to understand for predicting water resources and land-surface energy exchanges. Global changes in climate and vegetation cover have the potential to alter how snowfall is partitioned in forested basins around the world, however, the process understanding required to investigate how these changes may impact future water resources and land-surface energy exchanges is currently limited. Snowfall partitioning by needleleaf canopies regulates the quantity (estimates of over 50\%), timing, and phase of precipitation that reaches the ground in cold region forests. Sparse observations have limited understanding of processes governing forest-snow interactions and hindered their representation in hydrological and land-surface models. Consequently, simulations of the canopy snow mass and energy balance have shown variable accuracy across different climates and canopy structures, and thus diagnosing the causative processes that partition snowfall in needleleaf forests remains difficult. This study aims to better understand the processes underlying the canopy snow energy and mass balance to improve the accuracy of such models. To achieve this goal a comprehensive review of the literature was conducted, in-situ observations of forest-snow processes and subcanopy snow accumulation were collected to evaluate the theories underlying existing parameterisations, derive new relationships, and validate a newly proposed model. 

The primary gaps this research aims to address regarding snow interception and canopy snow ablation include, all existing theories of throughfall of snow in forests compute an increase in throughfall increasing as a function of canopy snow load. Determining whether this association of throughfall with canopy snowload is due to branch bending (decreasing canopy coverage) and/or increased rates of unloading remains uncertain.  In addition, many models of the canopy snow mass balance do not include representations of both dry snow unloading and melt-induced unloading. If they are both included, the melt-induced unloading component is typically based on a temperature index method. This identified research gaps on the need to assess if canopy snow unloading can be represented using a temperature index and dry snow unloading parameterisation or if more physically based methods are required. Moreover, the methodologies used to parameterise non-melt induced canopy snow unloading routines rely on indirect measurements that combine several different processes (unloading, melt, drip, and sublimation) and should be assessed using more direct measurements that enable process isolation. While many canopy snow ablation processes occur simultaneously using a hybrid modelled-observed processes to help isolate individual canopy snow processes has not been utilised to help understand canopy snow processes. Existing studies on snowfall partitioning by forest canopies have largely focused on forests with high density canopies and either cold continental or warm maritime climates. Another research gap is to investigate the applicability of existing theories in these environments across a diverse range in forest canopy densities and meteorologies. 

New observations of initial snow interception, collected when ablation processes were minimal, showed that canopy density is the primary factor influencing subcanopy snow throughfall. Contrary to existing theories, no relationships were found with canopy snow load or air temperature and initial interception. Canopy density was best represented by a snow-leaf contact area metric, which accounts for increasing contact area with more horizontal snowfall trajectories. This metric was highly sensitivity to wind speed in sparse canopies, increasing by up to an order of magnitude with a 1 m s$^{-1}$ wind speed showing that computations of throughfall in sparse canopies using binary canopy coverage data may fail. A new parameterisation was developed that calculates throughfall as a function of snowfall and snow-leaf contact area based on these observations improving simulation of throughfall across a range of canopy densities and meteorologies. This approach is consistent with some rainfall interception studies, which also separate canopy loading and ablation processes, and calculate interception as a function of canopy cover.

An evaluation of theories to represent canopy snow unloading using observations collected across a wide range in meteorological conditions showed snow load, wind shear stress, and canopy snow melt were the best predictors. These new relationships generally corroborated with previous studies on the broad processes represented (i.e., canopy load, wind, and melt), but provided increased physical basis by representing wind-driven unloading with shear stress and melt-induced unloading as a function of energy balance-based canopy snow melt. Moreover, the ratio of solid snow unloading to liquid meltwater during canopy snowmelt was found to be a function of snowload building on previous work which utilised a constant ratio. A new canopy snow ablation model was developed based on these associations and coupled with a novel canopy snow energy and mass balance in a hydrological model. The new model improved simulation of canopy snow ablation relative to previous approaches, especially for melt- and wind-dominated ablation events.

An uncalibrated validation of the revised canopy snow mass and energy balance conducted at four needleleaf forests characterised by differing tree species, canopy structure, and meteorological conditions resulted in reduced error in simulating subcanopy snow water equivalent (SWE) compared to an existing model. The largest reduction in error was observed for the warm-humid maritime sites and was attributed to better representation of the canopy snowmelt and melt-induced unloading process. Improved process representation also enabled clearer process diagnosis on the influence of vegetation on snow accumulation. At two cold, low-snowfall sites, roughly half of annual snowfall was lost to the atmosphere via sublimation of intercepted snow. A cold wind-exposed site with higher annual snowfall had increased unloading rates and reduced atmospheric losses. In contrast, at a temperate-maritime site, nearly half of annual snowfall melted within the canopy, and combined with melt-induced unloading led to the lowest fraction of snowfall lost through canopy snow sublimation.

This research reveals the variability in snowfall partitioning processes across diverse canopy densities and meteorologies and that this variability is not adequately represented in existing models. New parameterisations of the canopy snow mass and energy balance addressing these limitations are introduced to more explicitly represent the processes that govern snowfall partitioning by needleleaf canopies. This new approach led to more accurate simulations of canopy snow load and subcanopy SWE across a broad range of meteorologies and forest types and also improved process diagnosis, though further validation is required across a diverse range of sites.
}
