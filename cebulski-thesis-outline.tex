% Options for packages loaded elsewhere
\PassOptionsToPackage{unicode}{hyperref}
\PassOptionsToPackage{hyphens}{url}
\PassOptionsToPackage{dvipsnames,svgnames,x11names}{xcolor}
%
\documentclass[
  letterpaper,
  DIV=11,
  numbers=noendperiod]{scrartcl}

\usepackage{amsmath,amssymb}
\usepackage{iftex}
\ifPDFTeX
  \usepackage[T1]{fontenc}
  \usepackage[utf8]{inputenc}
  \usepackage{textcomp} % provide euro and other symbols
\else % if luatex or xetex
  \usepackage{unicode-math}
  \defaultfontfeatures{Scale=MatchLowercase}
  \defaultfontfeatures[\rmfamily]{Ligatures=TeX,Scale=1}
\fi
\usepackage{lmodern}
\ifPDFTeX\else  
    % xetex/luatex font selection
\fi
% Use upquote if available, for straight quotes in verbatim environments
\IfFileExists{upquote.sty}{\usepackage{upquote}}{}
\IfFileExists{microtype.sty}{% use microtype if available
  \usepackage[]{microtype}
  \UseMicrotypeSet[protrusion]{basicmath} % disable protrusion for tt fonts
}{}
\makeatletter
\@ifundefined{KOMAClassName}{% if non-KOMA class
  \IfFileExists{parskip.sty}{%
    \usepackage{parskip}
  }{% else
    \setlength{\parindent}{0pt}
    \setlength{\parskip}{6pt plus 2pt minus 1pt}}
}{% if KOMA class
  \KOMAoptions{parskip=half}}
\makeatother
\usepackage{xcolor}
\setlength{\emergencystretch}{3em} % prevent overfull lines
\setcounter{secnumdepth}{-\maxdimen} % remove section numbering
% Make \paragraph and \subparagraph free-standing
\ifx\paragraph\undefined\else
  \let\oldparagraph\paragraph
  \renewcommand{\paragraph}[1]{\oldparagraph{#1}\mbox{}}
\fi
\ifx\subparagraph\undefined\else
  \let\oldsubparagraph\subparagraph
  \renewcommand{\subparagraph}[1]{\oldsubparagraph{#1}\mbox{}}
\fi


\providecommand{\tightlist}{%
  \setlength{\itemsep}{0pt}\setlength{\parskip}{0pt}}\usepackage{longtable,booktabs,array}
\usepackage{calc} % for calculating minipage widths
% Correct order of tables after \paragraph or \subparagraph
\usepackage{etoolbox}
\makeatletter
\patchcmd\longtable{\par}{\if@noskipsec\mbox{}\fi\par}{}{}
\makeatother
% Allow footnotes in longtable head/foot
\IfFileExists{footnotehyper.sty}{\usepackage{footnotehyper}}{\usepackage{footnote}}
\makesavenoteenv{longtable}
\usepackage{graphicx}
\makeatletter
\def\maxwidth{\ifdim\Gin@nat@width>\linewidth\linewidth\else\Gin@nat@width\fi}
\def\maxheight{\ifdim\Gin@nat@height>\textheight\textheight\else\Gin@nat@height\fi}
\makeatother
% Scale images if necessary, so that they will not overflow the page
% margins by default, and it is still possible to overwrite the defaults
% using explicit options in \includegraphics[width, height, ...]{}
\setkeys{Gin}{width=\maxwidth,height=\maxheight,keepaspectratio}
% Set default figure placement to htbp
\makeatletter
\def\fps@figure{htbp}
\makeatother

\KOMAoption{captions}{tableheading}
\makeatletter
\@ifpackageloaded{caption}{}{\usepackage{caption}}
\AtBeginDocument{%
\ifdefined\contentsname
  \renewcommand*\contentsname{Table of contents}
\else
  \newcommand\contentsname{Table of contents}
\fi
\ifdefined\listfigurename
  \renewcommand*\listfigurename{List of Figures}
\else
  \newcommand\listfigurename{List of Figures}
\fi
\ifdefined\listtablename
  \renewcommand*\listtablename{List of Tables}
\else
  \newcommand\listtablename{List of Tables}
\fi
\ifdefined\figurename
  \renewcommand*\figurename{Figure}
\else
  \newcommand\figurename{Figure}
\fi
\ifdefined\tablename
  \renewcommand*\tablename{Table}
\else
  \newcommand\tablename{Table}
\fi
}
\@ifpackageloaded{float}{}{\usepackage{float}}
\floatstyle{ruled}
\@ifundefined{c@chapter}{\newfloat{codelisting}{h}{lop}}{\newfloat{codelisting}{h}{lop}[chapter]}
\floatname{codelisting}{Listing}
\newcommand*\listoflistings{\listof{codelisting}{List of Listings}}
\makeatother
\makeatletter
\makeatother
\makeatletter
\@ifpackageloaded{caption}{}{\usepackage{caption}}
\@ifpackageloaded{subcaption}{}{\usepackage{subcaption}}
\makeatother
\ifLuaTeX
  \usepackage{selnolig}  % disable illegal ligatures
\fi
\usepackage{bookmark}

\IfFileExists{xurl.sty}{\usepackage{xurl}}{} % add URL line breaks if available
\urlstyle{same} % disable monospaced font for URLs
\hypersetup{
  pdftitle={Advancing Snow Accumulation Models in Needleleaf Forests},
  pdfauthor={Alex Cebulski},
  colorlinks=true,
  linkcolor={blue},
  filecolor={Maroon},
  citecolor={Blue},
  urlcolor={Blue},
  pdfcreator={LaTeX via pandoc}}

\title{Advancing Snow Accumulation Models in Needleleaf Forests}
\author{Alex Cebulski}
\date{September 19, 2024}

\begin{document}
\maketitle

\subsection{Overview}\label{overview}

This document links proposed thesis objectives to thesis chapters and
proposed papers within each chapter.

\subsection{Thesis Summary}\label{thesis-summary}

Purpose: To better understand the processes that govern snow
accumulation in forested environments.

\subsubsection{Thesis objectives and research
questions:}\label{thesis-objectives-and-research-questions}

\begin{enumerate}
\def\labelenumi{\arabic{enumi}.}
\item
  Evaluate the suitability of existing snow interception and ablation
  parameterizations for application in needleleaf forests with differing
  canopy structure and meteorology.

  \begin{enumerate}
  \def\labelenumii{\alph{enumii}.}
  \item
    What are the theoretical underpinnings and assumptions behind
    existing snow interception and ablation parameterizations?
  \item
    Are the theories and assumptions of existing snow interception
    parameterizations supported by field measurements collected across
    diverse canopy structures and meteorological conditions?
  \item
    Are the theories and assumptions of existing canopy snow ablation
    parameterizations supported by field measurements collected across
    diverse canopy structures and meteorological conditions?
  \end{enumerate}
\item
  Quantify the performance of current snow interception
  parameterizations against field observations in differing forest
  structures and climates.

  \begin{itemize}
  \item
    For what climatic conditions and forest structures are predictions
    from current snow interception models most uncertain?
  \item
    How do the assumptions of existing snow interception
    parameterizations influence model performance?
  \end{itemize}
\item
  Determine how the modifications of existing snow interception
  parameterizations could improve process representations that are
  important for snow accumulation and redistribution in mountain
  forests.

  \begin{itemize}
  \item
    What is the change in simulated snow accumulation model error
    associated with revised canopy snow interception parameterizations
    across mountain forests of differing forest structure and climate?
  \item
    What is the change in simulated streamflow model error associated
    with revised canopy snow interception parameterizations across
    forested mountain basins of differing forest structure and climate?
  \end{itemize}
\end{enumerate}

\subsubsection{Organization of Chapters}\label{organization-of-chapters}

This thesis contains 4 chapters, the first chapter includes an
introduction and research plan while, the remaining chapters 2-4 each
correspond to a journal article which aims to answer each of the
research questions.

\subsection{Thesis Chapters}\label{thesis-chapters}

\subsubsection{Chapter 1}\label{chapter-1}

This chapter includes background information about the study. Including
the following subsections:

\paragraph{Introduction}\label{introduction}

This subsection of Chapter 1 will introduce background information,
motivation, hydrological significance of the study topic, research gaps,
and methods used in the study.

\paragraph{Research Plan}\label{research-plan}

This subsection of Chapter 1 will describe the overall purpose of the
thesis and links the individual thesis objectives to research gaps.

\subsubsection{Chapter 2}\label{chapter-2}

This chapter corresponds to objective 1 of the thesis, to evaluate the
suitability of existing snow interception parameterizations for
application in mountain forests with differing climate and forest
structure. To achieve this objective three draft journal articles have
been written:

\paragraph{Paper 1: The Theoretical Underpinnings of Existing Snow
Interception and Ablation
Parameterizations}\label{paper-1-the-theoretical-underpinnings-of-existing-snow-interception-and-ablation-parameterizations}

This paper aims to answer the first research question of Objective 1
which is ``What are the theoretical underpinnings and assumptions behind
existing snow interception parameterizations?''. This advanced review
will provide the context necessary for interpreting whether the theories
and assumptions of existing parameterizations are true for the field
observations collected in this study in the second part of objective 1.
This paper is an advanced review which has been invited for submission
to the journal WIREs Water.

\textbf{WIREs Water Deadline:} June 28, 2024

\textbf{Abstract:}

In needleleaf forests, up to half of annual snowfall may be lost due to
sublimation of snow intercepted in the canopy. However, a comprehensive
understanding of snow interception and ablation processes has been
constrained by a lack of observations. Existing parameterizations for
snow interception and ablation have been developed in locations with
distinct climate and forest structures, resulting in differing and
incomplete process representations. Consequently, their transferability
across diverse landscapes and climates remains uncertain. This review
article aims to elucidate the theoretical foundations and assumptions
underlying the current snow interception and ablation parameterizations
in the literature. The theory and methods behind snow interception and
ablation studies are also reviewed to provide the context necessary for
examining the applicability of current parameterizations across diverse
environments. Some gaps in the literature include challenges in
differentiating throughfall measurements from unloading and drip, the
assumption of vertical snowflake trajectories, the difficulty in
partitioning unloading rates and canopy snow melt drainage, the absence
of a wind resuspension parameterization, and the inadequate validation
of parameterizations in wind exposed subalpine forests. By reviewing the
theory, methods and assumptions of existing snow interception and
ablation parameterizations this article aims to inform future
model-decision makers in selecting appropriate parameterizations and
guiding future field-based observational studies.

\subparagraph{1. Introduction}\label{introduction-1}

\subparagraph{2. The Mass and Energy Balance of Snow in the
Canopy}\label{the-mass-and-energy-balance-of-snow-in-the-canopy}

Section 2 begins with discussing the symbology used to represent mass
fluxes, energy fluxes and states of snow intercepted in the canopy. This
section is written in the context of a winter needleleaf forest
environment.

2.1 Mass Balance

Section 2.1 contains a mass balance equation for canopy snow load
followed by a description of each of the terms in the equation. A figure
is also shown which gives a visual of the mass balance processes
important for canopy snow load. Coupled mass and energy equations for
the calculation of melt and sublimation of snow intercepted in the
canopy are also shown.

2.2 Energy Balance

Section 2.2 contains an energy balance equation for snow intercepted in
the canopy followed by a description of the terms. A figure showing a
visual representation of the energy balance processes is shown. A
discussion of some of the simplifications made with this energy balance
representation is then provided.

\subparagraph{3. Measuring Snow Interception and
Ablation}\label{measuring-snow-interception-and-ablation}

Section 3 covers the common methodologies used to measure snow
interception and ablation. A description of the principle behind each
method is given and any uncertainties related to the method are also
discussed. The methodologies described include, weighed tree, mass
balance methods, snow surveys, subcanopy lysimeters and remote sensing
techniques.

3.1 Weighed Tree

3.2 Mass Balance Methods

3.2.1 Snow Surveys

3.2.2 Subcanopy Lysimeters

3.3 Remote Sensing

\subparagraph{4. Methods of
Determination}\label{methods-of-determination}

Section 4 discusses the parameterizations available in the literature
for the determination of the mass and energy balance processes discussed
in Section 2. For each parameterization a description the study
environment, climate, and methodologies used to derive it is provided.
Section 4.1 discusses snow interception parameterizations followed by
section 4.2 which discusses snow ablation parameterizations for
sublimation, unloading and drip.

4.1 Snow Interception Parameterizations

4.2 Canopy Snow Ablation Parameterizations

4.2.1 Sublimation

4.2.2 Unloading and Drip

\subparagraph{5. Discussion}\label{discussion}

In Section 5 the theories and assumptions of the parameterizations
listed above are compared. Research gaps are also listed to give insight
on where current snow interception and ablation parameterizations
theories and assumptions may be invalid and where new observations and
theoretical development is required. Advice for informing model-decision
makers on choosing parameterizations is also given

\subparagraph{6. Conclusion}\label{conclusion}

\paragraph{Paper 2: Combined effects of wind, air temperature and
snowfall on snow interception in a subalpine
forest}\label{paper-2-combined-effects-of-wind-air-temperature-and-snowfall-on-snow-interception-in-a-subalpine-forest}

This journal article aims to answer part of the second research question
of Objective 1, ``Are the theories and assumptions of existing snow
interception parameterizations true for field measurements collected
across diverse forest structures and climates?''. This will be achieved
by presenting observations of interception from a study site few
researchers have focused on, a subalpine discontinuous forest and
contrast these results with existing theory developed in maritime and
continental climates. This journal article is in progress for submission
to the Hydrological Processes special issue ``Canadian Geophysical Union
2023''.

\subparagraph{1. Introduction}\label{introduction-2}

\subparagraph{2. Methods}\label{methods}

2.1 Study Site

2.2 Automated Interception Measurements

2.3 Snow Surveys

2.3.1 In-Situ Measurements

2.3.2 UAV-LiDAR Measurements

2.3.4 Discrete Event Interception Measurements

2.3 Canopy Structure Products

\subparagraph{3. Results}\label{results}

3.1 The influence of meteorology on snow interception

\begin{itemize}
\item
  The accumulation of canopy load over 26 snowfall events shown in
  Figure~\ref{fig-scl-w-sf} measured using the subcanopy lysimeters,
  exhibits the variability in I/P between and within the different
  events. The relatively low variability in I/P across and within the
  different events is attributed to variances in meteorological
  conditions.
\item
  Frequency distribution of meteorological variables observed over the
  26 snowfall events (\textbf{?@fig-hist-met-ip}).
\item
  \textbf{?@fig-lai-met-ip} shows 15-minute average variables including:
  air temperature, relative humidity, wind speed, initial canopy snow
  load, hydrometeor diameter, hydrometeor velocity, versus 15 minute
  average snow interception efficiency for all 26 snowfall events.
\end{itemize}

\begin{figure}

\begin{minipage}{0.50\linewidth}

\centering{

\captionsetup{labelsep=none}\includegraphics{../snow-int-paper/figs/automated_snowfall_event_periods/cuml_event_snowfall_canopy_storage_sep_scl.png}

}

\subcaption{\label{fig-scl-w-sf}}

\end{minipage}%
%
\begin{minipage}{0.50\linewidth}

\centering{

\captionsetup{labelsep=none}\includegraphics{../snow-int-paper/figs/automated_snowfall_event_periods/event_total_snowfall_vs_IP_colour_troughs.png}

}

\subcaption{\label{fig-scl-ip}}

\end{minipage}%

\caption{\label{fig-scl}Two plots showing the relationship between
snowfall and interception. Plot (a) shows the cumulative event snowfall
versus the corresponding state of canopy snow storage for each of the 26
snowfall events. Plot (b) shows total event snowfall versus the average
interception efficiency for each event. Snowfall data was measured using
the snowfall gauge at Powerline Station while throughfall data was
measured using the three subcanopy lysimeters used for the calculation
of canopy storage and interception efficiency. These lysimeters, each
denoted by a distinct color (black, red, and green), correspond to
varying canopy coverage (0.73, 0.78, and 0.82, respectively).}

\end{figure}%

\begin{figure}

\centering{

\includegraphics{../snow-int-paper/figs/automated_snowfall_event_periods/troughs_met_vs_IP_bin.png}

}

\caption{\label{fig-met-ip}Scatter plots of discrete observations
(green) of snow interception efficiency observed at 15 minute intervals
using the subcanopy lysimeter and snowfall gauge against and binned data
(black). Panels show (A) air temperature, (B) wind speed, (C) initial
canopy snow load (the snow load observed at the beginning of the
timestep), (E) hydrometeor diameter, (F) hydrometeor velocity. The black
open circles show the mean of each bin and the error bars represent the
standard deviations. The data were filtered to include observations with
a snowfall rate \textgreater{} 0 mm/hr and a snowfall rate
\textgreater{} the subcanopy lysimeter throughfall rate to minimize
observations with unloading. Periods of unloading and melt were also
removed through careful analysis of the weighed tree, subcanopy
lysimeters, and timelapse imagery.}

\end{figure}%

3.2 The influence of forest structure on snow accumulation

\begin{itemize}
\item
  Snow interception efficiency observed across the study site after a 24
  hour snow accumulation event reveals the influence of forest structure
  on snow accumulation.
\item
  The spatial distribution in I/P across the study site, calculated
  using throughfall from lidar measurements and snowfall from the Pluvio
  snowfall gauge is shown in Figure~\ref{fig-lidar-ip}. Greater I/P is
  observed on the north (lee) side of individual trees which is inferred
  to be due to the predominately southerly winds observed over this
  event. This effect is more apparent on the southwest of the study site
  compared to north and eastern locations within the study site.
\end{itemize}

\begin{figure}

\begin{minipage}{0.50\linewidth}

\centering{

\includegraphics{../../analysis/lidar-processing/figs/maps/pwl_e_23_072_23_073_v2.0.0_saip_normalised_resample_0.25.png}

}

\subcaption{\label{fig-lidar-ip-pwle}PWL I/P}

\end{minipage}%
%
\begin{minipage}{0.50\linewidth}

\centering{

\includegraphics{../../analysis/lidar-processing/figs/maps/fsr_s_23_072_23_073_v2.0.0_saip_normalised_resample_crop_0.25.png}

}

\subcaption{\label{fig-lidar-ip-fsrs}FT I/P}

\end{minipage}%

\caption{\label{fig-lidar-ip}Interception efficiency calculated over a
24 hr snowfall event from March 13, 2023 to March 14, 2024 for the PWL
forest plot (a) and FT forest plot (b) at a 25 cm resolution. White
areas for the bottom row are 25 cm grids that did not have any lidar
ground returns or have been masked due to disturbance and thus, no
throughfall measurement for the I/P calculation.}

\end{figure}%

\begin{itemize}
\tightlist
\item
  To determine how forest structure was associated with interception
  efficiency over the March 13-14 snowfall event, each portion of the
  hemisphere at each grid location was considered. The Spearman's
  Correlation Coefficient calculated between the single raster grid of
  I/P and multiple canopy contact number at a given portion of the
  hemisphere (azimuth {[}0, 1, \ldots, 359{]}, zenith angle {[}0, 1,
  \ldots, 90{]}) is shown in Figure~\ref{fig-hemi-ip-cc}.
\end{itemize}

\begin{figure}

\begin{minipage}{0.50\linewidth}

\centering{

\captionsetup{labelsep=none}\includegraphics{../../analysis/lidar-processing/figs/voxrs/hemis/cor_cn_ip/full_hemi_rho_p_cor_mcn_ip_23_072_vox_len_0.25m__gridgen_PWL_E.png}

}

\subcaption{\label{fig-hemi-cc-ip-pwle}}

\end{minipage}%
%
\begin{minipage}{0.50\linewidth}

\centering{

\captionsetup{labelsep=none}\includegraphics{../../analysis/lidar-processing/figs/voxrs/hemis/cor_cn_ip/full_hemi_rho_p_cor_mcn_ip_23_072_vox_len_0.25m__gridgen_FSR_S.png}

}

\subcaption{\label{fig-hemi-cc-ip-fsrs}}

\end{minipage}%

\caption{\label{fig-hemi-ip-cc}Spearman's Correlation Coefficient
between rasters of interception efficiency and canopy contact number (25
cm resolution) across the study site for different portions of the
hemisphere (azimuth {[}0, 1, \ldots, 359{]}, zenith angle {[}0, 1,
\ldots, 90{]}).}

\end{figure}%

3.3 Combined effects of Meteorology and Forest Structure

\begin{itemize}
\tightlist
\item
  The mean canopy contact number, obtained through voxel ray sampling
  across all azimuth angles {[}0, 1, \ldots, 359{]} for each zenith
  angle {[}0, 1, \ldots, 90{]}, shown in Figure~\ref{fig-ta-ws-cc}
  demonstrates an exponential rise in contact number with increasing
  (more horizontal) trajectory angle. This underscores the influence of
  hydrometeor trajectory angle, which is a function of wind speed, on
  the apparent forest structure important for interception.
\end{itemize}

\begin{figure}

\begin{minipage}{\linewidth}

\centering{

\captionsetup{labelsep=none}\includegraphics{../../analysis/lidar-processing/figs/voxrs/scatter/traj_angle_and_wind_vs_contact_number_phiby_2_thetaby_5.png}

}

\subcaption{\label{fig-ht-cc}}

\end{minipage}%
\newline
\begin{minipage}{\linewidth}

\centering{

\captionsetup{labelsep=none}\includegraphics{../../analysis/lidar-processing/figs/voxrs/scatter/traj_angle_and_wind_vs_canopy_coverage_phiby_2_thetaby_5.png}

}

\subcaption{\label{fig-ht-cc}}

\end{minipage}%

\caption{\label{fig-ta-ws-cc}Scatter plots showing the association of
(a) hydrometeor trajectory angle and wind speed with mean contact number
and (b) hydrometeor trajectory angle and wind speed with apparent canopy
coverage calculated as a function of mean contact number. The dots
represent the mean mean contact number (a) OR mean canopy coverage (b)
across all azimuth angles of the the hemisphere {[}0, 1, \ldots, 359{]}
for a each zenith angle {[}0, 1, \ldots, 90{]} at each forest plot. The
colour of the dot represents the mean canopy coverage of each forest
plot from nadir. Trajectory angle is calculated as zenith angle - 90°.}

\end{figure}%

\subparagraph{4. Discussion}\label{discussion-1}

\begin{itemize}
\tightlist
\item
  This discussion will aim to answer the second research question of
  objective 1, ``Are the theories and assumptions of existing snow
  interception parameterizations true for field measurements collected
  across diverse forest structures and climates?'' for a discontinuous
  subalpine forest. This will be achieved by comparing the theories of
  existing snow interception parameterizations reviewed in Paper 1 from
  maritime and continental climates with the observations presented in
  the results here.
\end{itemize}

\subparagraph{5. Conclusions}\label{conclusions}

\begin{itemize}
\item
  Forest structure is the main factor governing the fraction of
  intercepted snowfall at a particular site, with meteorological
  conditions contributing less to variability.
\item
  Figure~\ref{fig-scl}: Carefully selected snowfall events, prior to
  canopy snow ablation, did not approach a maximum snow load load, and
  interception efficiency was not observed to be associated with event
  size. This challenges existing snow interception parameterizations
  which rely on the assumption that interception efficiency is a
  function of maximum canopy snow load. While some rise in interception
  efficiency was observed alongside increasing canopy snow load,
  primarily attributed to increasing canopy coverage, the subsequent
  decrease in interception efficiency at higher loads implies that
  canopy snow ablation is proportional to canopy snow load.
\item
  Figure~\ref{fig-met-ip}: No influence of air temperature, relative
  humidity, hydrometeor velocity, hydrometeor diameter or canopy storage
  on interception efficiency was observed.
\item
  Figure~\ref{fig-met-ip}: Interception efficiency was shown to increase
  with wind speed and canopy snow load as a result of increasing
  snow-leaf contact area.
\item
  Figure~\ref{fig-met-ip}: High wind speeds were observed to decrease
  intercepted load due to increased snow unloading.
\item
  Figure~\ref{fig-lidar-ip}: Spatially distributed UAV-lidar
  measurements of throughfall from a snowfall event with steady wind
  shows reduced snow accumulation on the lee side of individual trees as
  a result of increasing snow-leaf contact area due to non vertical
  hydrometeors results.
\item
  A new snow interception parameterization has been presented which
  calculates initial interception, before canopy snow ablation, as a
  function of snowfall rate and snow-leaf contact area ratio.
\item
  A second new parameterization is proposed which calculates snow-leaf
  contact area ratio as a function of nadir canopy coverage, wind speed
  and canopy snow load.
\item
  Caution should be taken in using this updated interception routine
  with existing canopy snow ablation parameterizations as they were
  developed using earlier snow interception routines that also included
  ablative processes.
\item
  Future work will will involve a canopy snow ablation routine that is
  revised to work with this new snow interception routine.
\end{itemize}

\paragraph{Paper 3: The Impact of Meteorology on Canopy Snow Ablation
Processes: Insights from Field Observation in a Subalpine
Forest}\label{paper-3-the-impact-of-meteorology-on-canopy-snow-ablation-processes-insights-from-field-observation-in-a-subalpine-forest}

This journal article will present results of canopy snow ablation
observations from Fortress Mountain Research basin collected over the
2022 and 2023 water years. This journal article will follow a similar
story as in Paper 2 in that observations will be presented in the
results section and in the discussion question 2 of objective 1 will be
addressed. The results from this paper were presented at INARCH and AGU
2023.

\subparagraph{1. Introduction}\label{introduction-3}

Discuss the difficulty in obtaining canopy snow ablation measurements,
especially over space.

\subparagraph{2. Methods}\label{methods-1}

2.1 Study Site

2.2 Canopy Snow Ablation Measurements

2.3 Canopy Snow Sublimation Modelling (in absence of usable Eddy
Covariance data)

\subparagraph{3. Results}\label{results-1}

3.1 Dominant Ablation Processes Observed

\begin{itemize}
\item
  This first section of the results will present the apportionment of
  canopy snow ablation, shown in Figure~\ref{fig-c-abl}, determined
  using automated measurements of canopy snow ablation from the weighed
  tree and unloading from the subcanopy lysimeters
\item
  The apportionment of how canopy snow ablation changes with temperature
  in Figure~\ref{fig-c-abl-temp} and wind speed in
  Figure~\ref{fig-c-abl-wind} will also be presented. Also discuss 0.5
  mm/hr of canopy snow that was observed to be entrained above the
  canopy at wind speeds above 4 m/s.
\end{itemize}

\begin{figure}

\centering{

\includegraphics{../../conference/agu/2023-sanfran/R/figs/ablation/canopy_snow_ablation_partition_global.png}

}

\caption{\label{fig-c-abl}The apportionment of canopy snow ablation
determined using automated measurements of canopy snow ablation from the
weighed tree and unloading from the subcanopy lysimeters averaged over
two winter seasons. Sublimation was simulated using the Cold Regions
Hydrological Model (Pomeroy et al., 2007).}

\end{figure}%

\begin{figure}

\begin{minipage}{0.50\linewidth}

\centering{

\captionsetup{labelsep=none}\includegraphics{../../conference/agu/2023-sanfran/R/figs/ablation/canopy_snow_ablation_partition_vs_air_temp.png}

}

\subcaption{\label{fig-c-abl-temp}}

\end{minipage}%
%
\begin{minipage}{0.50\linewidth}

\centering{

\captionsetup{labelsep=none}\includegraphics{../../conference/agu/2023-sanfran/R/figs/ablation/canopy_snow_ablation_partition_vs_mid_canopy_wind.png}

}

\subcaption{\label{fig-c-abl-wind}}

\end{minipage}%

\caption{\label{fig-c-abl-met}The change in contribution of unloading
and the residual to total canopy ablation. Calculated using automated
measurements of ablation from the weighed tree and unloading from the
subcanopy lysimeters.}

\end{figure}%

3.2 The Influence of Meteorology on Unloading

\begin{itemize}
\item
  The probability of unloading shown in Figure~\ref{fig-prob-unl} was
  observed to be higher with air temperatures above 0 °C and with wind
  speeds above 2 m/s. At air temperatures above -6 °C the effect of wind
  speed on unloading appears to be reduced.
\item
  The observed unloading rate attributed to warming was higher at
  sub-zero temperatures when the canopy was loaded (\textgreater{} 6.5
  mm) compared to when there was less than 6.5 mm of snow in the canopy
  (Figure~\ref{fig-qunld-temp}).
\item
  High rates of unloading attributed to wind were observed across all
  wind speed bins when the canopy was loaded (\textgreater{} 6.5 mm).
  The unloading rate was observed to increase with increasing wind with
  less than 6.5 mm of snow in the canopy (Figure~\ref{fig-qunld-wind}).
\end{itemize}

\begin{figure}

\begin{minipage}{0.50\linewidth}

\centering{

\captionsetup{labelsep=none}\includegraphics{../../conference/agu/2023-sanfran/R/figs/ablation-binned/probability_of_unloading_wind_air_temp.png}

}

\subcaption{\label{fig-prob-unl}}

\end{minipage}%
%
\begin{minipage}{0.50\linewidth}

\centering{

\captionsetup{labelsep=none}\includegraphics{../../conference/agu/2023-sanfran/R/figs/ablation-binned/frequency_of_unloading_wind_air_temp.png}

}

\subcaption{\label{fig-prob-freq}}

\end{minipage}%

\caption{\label{fig-bin-unl}The probability and frequency of unloading
for air temperature and wind speed bins pairs measured using automated
measurements of unloading from the subcanopy lysimeters. The probability
of unloading was calculated as the number of unloading events within
each bin pair divided by the total number of occurrences of each bin
pair.}

\end{figure}%

\begin{figure}

\begin{minipage}{\linewidth}

\centering{

\captionsetup{labelsep=none}\includegraphics{../../conference/agu/2023-sanfran/R/figs/ablation-binned/binned_unloading_rate_and_t_air_class_canopy_load.png}

}

\subcaption{\label{fig-qunld-temp}}

\end{minipage}%
\newline
\begin{minipage}{\linewidth}

\centering{

\captionsetup{labelsep=none}\includegraphics{../../conference/agu/2023-sanfran/R/figs/ablation-binned/binned_unloading_rate_and_wind_mid_class_tree_load.png}

}

\subcaption{\label{fig-qunld-wind}}

\end{minipage}%

\caption{\label{fig-qunl-met}Average unloading rates measured by the
subcanopy lysimeters for periods where wind speeds are less than 2 m/s
(a) OR air temperature less than -6 °C (b). Uncertainty ranges shows the
5th and 95th percentiles.}

\end{figure}%

\subparagraph{4. Discussion}\label{discussion-2}

This discussion will aim to answer the second research question of
objective 1, ``Are the theories and assumptions of existing snow
interception parameterizations true for field measurements collected
across diverse forest structures and climates?'' for a discontinuous
subalpine forest. This will be achieved by comparing the theories of
existing canopy snow ablation parameterizations reviewed in Paper 1 from
maritime and continental climates with the observations presented in the
results here.

\subparagraph{5. Conclusions}\label{conclusions-1}

\subsubsection{Chapter 3}\label{chapter-3}

This chapter corresponds to objective 2, to quantify the performance of
current snow interception parameterizations against field observations
in differing forest structures and climates. To achieve this objective
one journal article is proposed:

\paragraph{Paper 4: The Influence of Climate and Forest Structure on
Snow Interception Parameterization Performance: Insights for Improved
Process Representation in Mountain
Forests}\label{paper-4-the-influence-of-climate-and-forest-structure-on-snow-interception-parameterization-performance-insights-for-improved-process-representation-in-mountain-forests}

The content of this paper is still to be discussed with John.

The plan in the thesis proposal was to evaluate the CRHM canopy modules
using measurements of sub-canopy SWE, interception and ablation measured
at various spatial and temporal scales at Fortress Mountain, Marmot
Creek, Wolf Creek, and Russell Creek. A limitation of this approach is
the limited spatial coverage in continuous point scale process
measurements (weighed tree, lysimeters), also doesn't answer forest
structure component of the objective. The advantage of this plan is more
detailed process investigation (also assuming better weighed tree was
installed fall of 2022 at Wolf Creek compared to the dead tree hung in
2021).

While not in the original proposal, after spending a week learning CHM
with Chris Marsh I thought a similar analysis could be conducted using
CHM by updating the canopy module and comparing simulated SWE to aerial
lidar SWE across Fortress Basin and Russell Creek. This could be
achieved using the monthly Fortress basin and Vancouver Island aerial
lidar snow depth measurements. The disadvantage of this plan would be
less detailed process investigation. The advantage is greater spatial
coverage across variable forest structure and still contrasting climates
(Fortress vs.~Vancouver Island).

\subsubsection{Chapter 4}\label{chapter-4}

This chapter corresponds to objective 3, determine how the modification
of existing snow interception parameterizations better represent the
processes important for snow accumulation and redistribution in mountain
forests of differing structure and climate

\paragraph{Paper 5: An evaluation of new snow interception and ablation
parameterizations across diverse forest structures and
climates}\label{paper-5-an-evaluation-of-new-snow-interception-and-ablation-parameterizations-across-diverse-forest-structures-and-climates}

As in chapter 3, the content of this paper is still to be discussed with
John.

The proposed content of this paper aims to answer the first research
question of objective 3, what is the change in forest snow accumulation
model error associated with an updated canopy snow interception
parameterization?. To achieve this, insights gained from objective 1 and
2 will be used to inform the modification of existing snow interception
parameterizations. The updated parameterizations will be evaluated by
including them in an updated CRHM canopy module and compared simulated
SWE to observed SWE within the forested portion of each basin. As in
paper 4 this could also be conducted using CHM with implications for
changing the analysis detail and spatial scaling.

\paragraph{Paper 6: TBD}\label{paper-6-tbd}

To answer the second research question of objective 3, ``What is the
change in forested basin streamflow model error associated with an
updated canopy snow interception parameterization?'', a sixth paper
would be required.

This paper would involve setting up CRHM (CHM does not have stream
routing yet) for mountain basins with a high fraction of their
snow-covered zone that is forested and running a sensitivity analysis on
predicted streamflow using different versions of the CRHMs canopy
module. Possible test basins that meet this criteria are: the Tsitika
River (Coastal, 08HF004), Upper Penticton River (Interior Dry, 08NM240),
Kuskanax Creek (Interior Wet, 08NE006), and White Gull Creek
(Continental Cold Dry, 05KE010).

Further discussion with John should occur to ensure this paper is still
justified. After reflection while creating and teaching the Runoff
Module for GEOG 225, I realize that the uncertainty in the
snowmelt-runoff portion of hydrological models may mask any sensitivity
to changes to canopy module.



\end{document}
